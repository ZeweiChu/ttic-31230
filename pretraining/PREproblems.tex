\documentclass{article}
\input ../preamble
\parindent = 0em

%\newcommand{\solution}[1]{}
\newcommand{\solution}[1]{\bigskip {\color{red} {\bf Solution}: #1}}

\begin{document}


\centerline{\bf TTIC 31230 Fundamentals of Deep Learning}
\bigskip
\centerline{\bf Transformer Problems.}

\bigskip
\bigskip
{\bf Problem 1.} The Transformer computes layers of sequences of vectors $M[\ell,t,j]$ where
$\ell$ ranger over layers, $t$ ranges over ``time'' (the index into the sequence), and $j$ is the index
of a particular dimension of the vector at layer $\ell$ and time $t$.

\medskip
We also let $h$ range over ``heads'' --- each head is computing an attention for a different purpose.

\medskip
We compute compute $M[\ell+1,T,J]$ from $M[\ell,T,J]$ as follows:
      
\begin{eqnarray*}
\mathrm{Query}[\ell,h,t,{\color{red} k}] & = & \sum_j\;{\color{red} W^Q}[\ell,h,{\color{red} k,j}]M[\ell,t,{\color{red} j}] \\
\\
\mathrm{Key}[\ell,h,t,{\color{red} k}] & = & \sum_j\;{\color{red} W^K}[\ell,h,{\color{red} k,j}]M[\ell,t,{\color{red} j}] \\
\\
\mathrm{Value}[\ell,h,t,{\color{red} i}] & = & \sum_j\;{\color{red} W^V}[\ell,h,{\color{red} i,j}]M[\ell,t,{\color{red} j}]
\\
s[\ell,h,t,t'] & = &\frac{1}{\sqrt{K}} \sum_k\;\mathrm{Query}[\ell,h,t,k]\mathrm{Key}[\ell,h,t',k] \\
\\
{\color{red} \alpha[\ell,h,t,t']} & {\color{red} =} & {\color{red} \softmax_{t'} \;s[\ell,h,t,t']}\;\;\;\mbox{the attention} \\
\\
V[\ell,h,t,i] & = & \sum_{t'}\;\alpha[\ell,h,t,t']\mathrm{Value}[\ell,h,t',i]
\end{eqnarray*}


$$M[\ell+1,t,J] = V[\ell,1,t,I];V[\ell,2,t,I];\cdots;V[\ell,H,t,I] \;\;\;\; J = HI$$

(a) assume that summations are done serially on a GPU while other computations are maximally parallel.  Under these assumptions what is the order of the parallel running time
of the Transformer as a function of $L$, $T$, $H$, $J$ and $K$ (we have $I = J/H$).

\solution{
  The computation must be serial over the layers but we need to determine the parallel running time for computing $M[\ell+1,T,J]$ from $M[\ell,T,J]$
  The first three equations can be computed in parallel over $t$, $h$, $k$ and $i$. We have assumed that each summation takes time proportional to $J$.
  The fourth equation can be computed in parallel over $h$, $t$ and $t'$ and takes time proportional to $K$.  The softmax operation can be parallelized over
  $h$ and $t$ but requires computing the normalizing
  constant $Z$ by summing over $t'$ and takes time proportional to $T$.  Computing $V[\ell,h,t,i]$ can be parallelized over $h$, $t$ and $i$ but sums over $t'$
  and hence takes time proportional to $T$.  The concatenation operation can be fully parallelized and so takes unit time in parallel.  Putting this all together
  we get
  $$O(L(J + K + T))$$
}

(b) Actually a summation over $N$ terms can be done in parallel in $O(\log N)$ time.  Repeat part (a) but under the assumption that the summations take logarithmic parallel time.

\solution{
  $O(L(\log J + \log K + \log T))$

  \medskip
  The Transformer is much faster than an RNN.  It is comparable in speed to a CNN but with each position in each layer taking input from all positions in the previous layer.
}

\bigskip

{\bf Problem 2.}
Just as CNNs can be done in two dimensions for vision and in one dimension for language, the Transformer can be done in two dimensions for vision --- the so-called spatial transformer.

\medskip
(a) Rewrite the equations from problem 1 so that the time index $t$ is replaced by spatial dimensions $x$ and $y$.

\medskip
(b) Assuming that summations take logarithmic parallel time, give the parallel order of run time for the spatial transformer as a function of $L$, $X$, $Y$, $H$, $J$, and $K$.

\end{document}
