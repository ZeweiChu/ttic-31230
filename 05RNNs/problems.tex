\documentclass{article}
\input ../preamble
\parindent = 0em

\begin{document}


\centerline{\bf TTIC 31230 Fundamentals of Deep Learning}
\medskip
\centerline{\bf Problems For Language Modeling, Translation and Attention.}

\bigskip
\bigskip
{\bf Problem 1.} What is the order of the number of floating point operations (serial computer running time)
as a function of input sentence length and output sentence length
for both the forward and backward pass of sequence to sequence models for machine translation both with and without attention.

\bigskip
{\bf Problem 2.} Consider a bidirectional RNN run on a sequence of words $w_1,\ldots,w_T$ such that for each time $t$ we have a forward
hidden state $\vec{h}[t,J]$ computed from $w_1,\ldots,w_t$ and a backward hidden state $\cev{h}[t,J]$ computed from $w_T,\;w_{T-1},\ldots w_t$.

\medskip
(a) Given an explicit index (Einstein notation) definition of a cross entropy loss ${\cal L}_t$ for $P(w[t]\;|\;w_1,\ldots w_{t-1},\;w_{t+1},\ldots,w_T)$ as a function of
$\vec{h}[t-1,J]$ and $\cev{h}[t+1,J]$. You should define
the probability with a softmax and assume that softmax is given as a primitive.  Assume a word embedding matrix $e[W,J]$ where $e[w,J]$ is the embedding vector for word $w$.

\medskip
(b) Suppose we take the loss of a given model on a sentence $w_1,\ldots,w_t$ to be $\sum_t {\cal L}_t$ for ${\cal L}_t$ defined as in part (a).  What is the order
of run time, as a function of sentence length, for the backpropagation with this loss function?  Explain your answer.

\end{document}
